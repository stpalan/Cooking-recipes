
%------------------------------------------
% information doc
\title{Chocolate chip cookies}
\Preamble{\noindent This recipe stems from Nagi Maehashi at \burl{https://www.recipetineats.com/the-chocolate-chip-cookies-of-my-dreams}.}
\PrepTime{45}
\CookingTime{17}
\CookingTempe{170}
\TypeCooking{Oven -- Convection}
\NbPerson{8} %Number of cookies in this case
\WtPerPiece{150} %Weight in g per cookie
\Image{0 100 400 0}{images/florentin} %style 1 and 3
%\Image{0 100 400 200}{images/florentin} %style 1 and 3
%\Image{0 100 800 900}{images/florentin} %style 2
%------------------------------------------

\begin{ingredient}
%\vspace{0.5cm}
\begin{main}
	\item 225\,g butter
\end{main}
\begin{subingredient_blue}{Dry ingredients}
	\item 300\,g plain flour (typically wheat)
	\item 3\,tsp cornstarch
	\item 1/4\,tsp baking powder
    \item 1/2\,tsp baking soda
	\item 1/2\,tsp table salt	
\end{subingredient_blue}
\begin{subingredient_green}{Wet ingredients}
	\item 200\,g brown sugar
	\item 100\,g caster sugar (powdered sugar)
	\item 1 large egg (55\,g in shell), at room temp
    \item 1 yolk from a large egg, at room temp
    \item 1\,tsp vanilla extract
\end{subingredient_green}
\begin{subingredient_ochre}{Chocolate}
	\item 250\,g dark chocolate (70\,\%)
    \item 150\,g milk chocolate
\end{subingredient_ochre}
\end{ingredient} %no space with \begin{recipe}
\begin{recipe}

\mainstep{Browned butter:}
\substep{Put the butter in a silver saucepan over medium high heat (6 on my stove). Once melted, let it simmer (as in, bubbling) for 4 to 5 minutes, stirring every now and then, until it gets real foamy, you see little golden specks (wade through foam to see) and it smells nutty and extra buttery.}
\substep{Immediately pour into a heatproof bowl, including all those golden specks. Cool to room temperature (\~45 minutes), cool enough so it won't melt the sugar or chocolate when mixed with them.}

\mainstep{Ingredient preparation:}
\substep{Whisk {\color{ingredientblue}Dry ingredients} in a separate bowl.}
\substep{Cut the {\color{ingredientochre}Chocolate} into roughly 5\,mm~$\times$~5\,mm pieces using a large knife.}

\mainstep{Dough:}
\substep{Mix {\color{ingredientgreen}Wet ingredients}: to the browned butter, add both sugars and mix with a wooden spoon. Add the egg, yolk and vanilla. Mix until smooth -- it will look caramel.}
\substep{Add the {\color{ingredientblue}Dry ingredients} and mix until the flour is mostly incorporated. Add the {\color{ingredientochre}Chocolate} and stir until the flour is fully incorporated.}
\substep{Make dough discs -- Measure out 10 x 120g portions of dough, roll into a ball then shape into a 3.5\,cm thick round discs. Place in a very airtight container.}

\mainstep{Chilling:}
\substep{Refrigerate for 12 hours and up to 48 hours.}

\mainstep{Baking:}
\substep{Preheat oven to 170°C fan-forced (or 180°C top/bottom heat).}
\substep{Place fridge-cold cookies 6\,cm apart on a tray lined with baking paper.}
\substep{Bake for 17 minutes or until the edges are golden and the surface is just set but still pale.}
\substep{Cool on the tray for 20 minutes – finishes baking, edges crisp more and they get more golden all over.}
\substep{Store in an airtight container. They are excellent for 2 days, still near excellent on day 3, still great on days 4 and 5.}

\end{recipe}

\begin{notes}
\begin{itemize}
\item Recipe can be scaled up as desired. If doubling, you can use 3 whole eggs instead of 2 whole eggs + 2 yolks. Suggest using stand mixer as dough gets quite thick towards end!
\item Baking soda (bi-carb): don't substitute with more baking powder if you don't have baking soda. You need baking soda!
\item Chilling:
\begin{itemize}
    \item 5\,h: Absolute minimum.
    \item 8\,h: Target minimum, deemed ``company worthy''.
    \item 12\,h -- 48\,h: Recommended.
    \item 3 -- 5 days: Diminishing returns, better to freeze at 12 hours, then thaw on demand.
\end{itemize}
\item Freezing: Uncooked dough discs can be frozen after the 12 hour fridge time. Bake from frozen per recipe (cookies are a bit thicker), or I prefer to thaw overnight in the fridge than bake per recipe.
\end{itemize}
\end{notes}	